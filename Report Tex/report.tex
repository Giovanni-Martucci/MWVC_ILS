\documentclass[11pt, oneside]{article} 
\usepackage{graphicx}
\graphicspath{ {./images/} }

\usepackage{algorithm}
\usepackage{algpseudocode}
\usepackage{hyperref}

\usepackage{sectsty}
\usepackage{graphicx}
\usepackage[font=small,labelfont=bf]{caption} % Required for specifying captions to tables and figures

\usepackage{color}
\pagecolor{white}

% Margins
\topmargin=-0.45in
\evensidemargin=0in
\oddsidemargin=0in
\textwidth=6.5in
\textheight=9.0in
\headsep=0.25in


\title{ %
\includegraphics[width=0.4\textwidth]{logo.png}~\\

\qquad


\noindent\rule{17cm}{0.4pt}

\qquad

Minimum Weight Vertex Cover with Iterated Local Search. \\ 

\qquad

\qquad

\large Laboratorio Intelligenza Artificiale (LM-18) \\ Università degli Studi di Catania - A.A 2021/2022 \\

\qquad

\qquad

\noindent\rule{17cm}{0.4pt}

\qquad

\qquad

}



\author{ Martucci Giovanni}
\date{\today}

\begin{document}

\maketitle	
%SetFonts

%\section{}
%\subsection{}

\pagebreak

% Optional TOC
% \tableofcontents
% \pagebreak

%--Paper--

\section{Introduction}

The aim of the project was to study and implement the Iterated Local Search to solve the problem of the Minimum Weight Vertex Cover belonging to the group of problems NP-Complete, therefore unsolvable with deterministic algorithms. A local search starts from a valid solution and iteratively builds a new solution trying to improve it until reaching a global optimal. During the execution of the algorithm the solution may end up in a very good local. Thanks to the perturbation operators and the acceptance criteria, the solution of local optimal is then removed by exploring new areas of the research space. 

\qquad
\qquad 

The entire project is available at the following Github repository:  \href{https://github.com/Giovanni-Martucci/MWVC_ILS}{https://github.com/Giovanni-Martucci/MWVC-ILS}.

\qquad

\qquad

\qquad

\section{Implementing rules}

\qquad

\qquad

\subsection{Generation of initial solution}
The first phase of the algorithm involves the generation of the initial solution following a Greedy logic. The first version of the solution provided for the insertion of the vertex with lower weight for each arc; subsequently, as shown in the algorithm in figure 1, a queue is used to track nodes that must be entered to meet the MWVC constraint. This priority queue is sorted by the degree ratio of each node divided by its weight. (degree[i]/weight[i]) so as to get the nodes with large degree and low weight above the queue.
\qquad


\begin{algorithm}
    \caption{ \texttt{Greedy Initial Solution}}
    \begin{algorithmic}
	    \Require {\texttt{graph, solution}}
	    \State{\texttt{E = outsideEdges(solution)}}
        	    \State{\texttt{queue = buildingQueue(E, graph)}}

	     \While{\texttt{E until 0}}
        		\State \texttt{topNode = queue.pop()}
        		\State \texttt{Add topNode to solution}
        		\State \texttt{Update E \textbackslash  edges(topNode)}
        		\State \texttt{Update queue}

            \EndWhile
    \end{algorithmic}
    
\end{algorithm}
\qquad

Then some useful parameters have been initialized and set for the convergence of the algorithm: the \textit{Epsilon}, which will be used in the Acceptance Criteria discussed later and the \textit{maximum number of bits} that will be perturbed in the current solution. In both cases the two parameters have an initial value and as the algorithm refines the solution they gradually decrease in such a way as to not allow a large change of the solution when you are close to the global optimal. Several solutions have been tried regarding the calibration of these parameters. In the source code it is possible to view the different implementations made before arriving at the final solution for both parameters, shown in paragraph 2.2.

\qquad 

The solution is represented as a list of values 1 or 0 indicating the presence or not of the i-th node in the final solution.

\qquad

\qquad

\qquad

\subsection{Parameters scheduling}

\qquad 

\qquad 

Below are the final solutions adopted for the parameters just discussed.

\qquad

\qquad


\subsubsection{Eps-scheduling}

\qquad

\qquad

   
\begin{algorithm}
    \caption{\texttt{Eps-scheduling}}
    \begin{algorithmic}
        \Require {\texttt{currentWeight, bestWeight, eps, minEps,  iterWithoutImprovs, maxIterWithoutImprovs}}
      	\If{\texttt{currentWeight < bestWeight}}
             \State{\texttt{epsNew = epsNew - ((bestWeights - currentWeights) * 2)}}
	        \If{\texttt{epsNew < minEps}}
        		     \Return { \texttt{minEps}}
		\EndIf{}
	 
        \Else{\texttt{}}

       \State{\texttt{epsNew = eps}}
         \If{\texttt{currentIter \textbf{mod} 6000 == 0 \textbf{and} iterWithoutImprovs > maxIterWithoutImprovs}}
        		     \State{\texttt{less = epsNew / 3}}
	        	     \State{\texttt{epsNew = epsNew - less}}
         \EndIf{}
          \If{\texttt{epsNew < 20}}
        		    \State{\texttt{epsNew = 100 * rand(1,4)}}
		\EndIf{}
        \EndIf{}
    \Return { \texttt{epsNew}}
    \end{algorithmic}
    \end{algorithm}

\qquad

As shown in figure 2 the value of Epsilon decreases with every improvement of the solution in such a way as to decrease the range of acceptance of the future iterations. Subsequently, with each number of fixed iterations X (in this case 6000), if the number of iterations without improvement exceeds the maximum limit (fixed at a certain arbitrary value), the value of Epsilon decreases by 1/3; Finally, if Epsilon becomes smaller than 20 it's returned to a value equal to 100 * random value from 1 to 4.




\pagebreak

\qquad

\subsubsection{Num-elem-to-change}



This parameter is used to iteratively decide how many values of the current solution you need to change. As the solution improves the following parameter decreases. Finally, if the number of iterations without improvements reaches a large arbitrary value (fixed at 150000) and the current number of iterations is less than 80\% of the maximum of available iterations, the number of items to change increases, so try to get out of a great local situation in case you are there.

\qquad

\begin{algorithm}
    \caption{\texttt{changeNumberElem}}
    \begin{algorithmic}
        \Require {\texttt{eps, minEps, maxEps, pertMax, pertMin}}
        \State{\texttt{improvs = 1.0 - ((eps - minEps) / (maxEps - minEps))}}
        \State{\texttt{newNumElemsToChange = pertMax - (improvs * (pertMax - pertMin))}}
        \If{\texttt{iterWithoutImprovements \textbf{mod} 15000 == 0 \textbf{and} currentIter < (maxIter - (maxIter / 20))}}
        		    \State{\texttt{newNumElemsToChange = numTotalNodes / 2}}
		\EndIf{}
    \Return { \texttt{newNumElemsToChange}}
    \end{algorithmic}
    \end{algorithm}


\qquad

\qquad

\qquad


\subsection{ Perturbation of the solution}

The general idea of perturbation is to modify the bits present in the solution at each iteration by applying a perturbation on the selected nodes. This procedure will allow you to move between the space of the solutions. It should be noted that a large perturbation creates a large displacement in the space of the solutions, whereas a small perturbation shifts the current solution into the solutions close to them. At first, in the first strategy developed only one random node is subtracted from the solution and another one is added, while if all nodes are present in the solution, only one random node is subtracted. Below is the pseudo-code of the final logic adopted for the perturbation, in which it comes prefixed the number of bits to change:
\qquad

\qquad

\begin{algorithm}
    \caption{\texttt{Perturbation}}
    \begin{algorithmic}
        \Require {\texttt{elemToChange, solution}}
        \For{\texttt{1 to elemToChange}}
       	 \State \texttt{v = randomNode(V)}
	 \State \texttt{Insert or Remove v inside solution}
     	\EndFor{}
    \Return { \texttt{solution}}
    \end{algorithmic}
    \end{algorithm}

The number of changes that will be made is defined by the parameter taken in input "numElemToChange". This parameter gradually decreases when you are close to the global optimum.
After having carried out the perturbation, we move on to the Local Search phase.

\pagebreak


\subsection{Validity of the solution}

\qquad

\qquad


After the solution has been perturbated, the solution is checked for validity. This is because a solution is valid if for each arc of the graph there is at least one of the two vertices inside the final solution. This check is shown in Figure 5:

\qquad

\qquad

\begin{algorithm}
    \caption{\texttt{Valid Solution}}
    \begin{algorithmic}
        \Require {\texttt{solution}}
        \For{\texttt{u in V}}
       		 \For{\texttt{(u,v) in adjList(u)}}
		 	 \If{\texttt{ $u \not\in Solution$  \textbf{and}  $v \not\in solution $}}
        		   		\Return {\texttt{False}}
			\EndIf{}
     		\EndFor{}
	\EndFor{}
    \Return { \texttt{True}}
    \end{algorithmic}
    \end{algorithm}


\qquad

As the figure shows, scrolling through all the adjacency lists for each node, at least one of the vertex nodes for each arc must be contained in the final solution.

\qquad

\qquad

\qquad

\qquad

\qquad


\subsection{Local search}

Before starting with this phase, the validity of the solution is always checked, using the algorithm discussed above. If the solution is not valid, scroll through the list of arcs whose both vertices do not belong to the solution and insert one of the two nodes inside the solution, by comparing the two based on the highest degree number and the lowest weight number. You can see the code in image 6:


\qquad

\qquad

\qquad

\qquad

\pagebreak

\begin{algorithm}
    \caption{\texttt{Local Search}}
    \begin{algorithmic}
        \For{\texttt{1,2,..., numSwaps}}
       		 \State \texttt{v1 = randomNode(solution)}
		  \If{\texttt{ $v1 \in Solution$}}
        		   		\State \texttt{v2 = randomNode(N(v1) \textbackslash  solution)}
				\If{\texttt{ weight(v2) < weight(v1)}}
        		   			\State \texttt{neighbor = solution - v1 + v2)}
				\EndIf{}
		 
		  \Else{\texttt{}}
			\State \texttt{$v2 = randomNode(N(v1) \in  solution)$}
			\If{\texttt{ weight(v1) < weight(v2)}}
        		   			\State \texttt{neighbor = solution - v2 + v1}
				\EndIf{}
			
		 \EndIf{}
		  \If{\texttt{neighbor is valid solution and weight(neighbor) < weight(currentSolution)}}
        		   			\State \texttt{ currentSolution = neighbor}
		  \EndIf{}	

	\EndFor{}
    \Return { \texttt{currentSolution}}
    \end{algorithmic}
    \end{algorithm}

\qquad

\qquad

\qquad


The implemented local search allows to complete the perturbed solution. You select a random node from the current solution and then a random neighbor is selected from its list of adjacencies that will replace it. If the neighbour will not bring any improvement then it will be passed to the selection of another node to replace, proceeding so iteratively. If instead the randomly selected node to be replaced is not part of the solution then you will search from its list of neighbors a node contained in the solution to be deleted to make room for the one initially selected.



\pagebreak

\subsection{Acceptance Criteria}

\qquad

\qquad


Acceptance criteria are necessary to assess the quality of the solution obtained from a local search before it is perturbed and used in the next iteration. In detail a solution is accepted if:

-       is better than the previous solution;

\qquad

-	the weight of the current solution is less than the optimal solution added to an Epsilon value and the number of iterations without improvement exceeds the maximum threshold of iterations without improvement;


\qquad

\qquad


\begin{algorithm}
    \caption{\texttt{Acceptance Criteria}}
    \begin{algorithmic}
     \If{\texttt{currentWeight < prevWeight \textbf{or} currentWeight < (bestWeight + eps) \textbf{and} iterWithoutImprovs > maxIterWithoutImprovs}}
        		   			\Return { \texttt{True}}	
	\Else{}
	\Return { \texttt{False}}	
 \EndIf{}
            
    \end{algorithmic}
    \end{algorithm}



\pagebreak


\section{Benchmarks}

The algorithm developed is able to find an excellent solution for each input, the introduction of the parameter \textit{Epsilon} and \textit{numElemToChange} has helped to achieve a fairly comprehensive neighborhood discovery greatly improving the final results.

\qquad

The following are the benchmarks: 


\begin{center}
\begin{minipage}{1.0\linewidth}
\includegraphics[width=\linewidth]{200_750_01.png}
\end{minipage}%

\qquad

\qquad

\begin{minipage}{1.0\linewidth}
\includegraphics[width=\linewidth]{20_120_01.png}
\end{minipage}

\qquad

\qquad

\begin{minipage}{1.0\linewidth}
\includegraphics[width=\linewidth]{25_150_01.png}
\end{minipage}

\qquad

\qquad

\begin{minipage}{1.0\linewidth}
\includegraphics[width=\linewidth]{100_500_01.png}
\end{minipage}

\qquad

\qquad

\begin{minipage}{1.0\linewidth}
\includegraphics[width=\linewidth]{100_2000_01.png}
\end{minipage}

\qquad

\qquad

\begin{minipage}{1.0\linewidth}
\includegraphics[width=\linewidth]{20_60_06.png}
\end{minipage}

\qquad

\qquad

\captionof{figure}{ Grafici di convergenza}
\end{center}


\begin{table}[]
\begin{tabular}{|l|l|l|l|l|l|}
Input name            & Initial Weight & Final Weight & Final Solution         & Final Eps   & Run Time    \\ \hline
vc\_20\_60\_01.txt    & 861            & 774          & {[}1, 2, 3,..., 18{]}  & 127,5555556 & 2,669713974 \\ \hline
vc\_20\_60\_02.txt    & 1026           & 938          & {[}0, 3, 6,...,19{]}   & 152         & 2,684123039 \\ \hline
vc\_20\_60\_03.txt    & 796            & 730          & {[}0, 3, 5,...,19{]}   & 117,9259259 & 2,296877146 \\ \hline
vc\_20\_60\_04.txt    & 819            & 769          & {[}0, 1, 2,...,18{]}   & 121,3333333 & 2,56937027  \\ \hline
vc\_20\_60\_05.txt    & 926            & 871          & {[}1, 3, 4,..., 19{]}  & 137,1851852 & 2,281625748 \\ \hline
vc\_20\_60\_06.txt    & 890            & 855          & {[}0, 2, 3,...,18{]}   & 131,8518519 & 2,445345163 \\ \hline
vc\_20\_60\_07.txt    & 1006           & 984          & {[}0, 1, 6,...,18{]}   & 149,037037  & 2,359009027 \\ \hline
vc\_20\_60\_08.txt    & 895            & 867          & {[}2, 4, 5,...,19{]}   & 132,5925926 & 2,198531866 \\ \hline
vc\_20\_60\_09.txt    & 1002           & 980          & {[}1, 3, 5,...,19{]}   & 148,4444444 & 2,45902276  \\ \hline
vc\_20\_60\_10.txt    & 875            & 875          & {[}3, 4, 6,...,19{]}   & 129,6296296 & 2,339637041 \\ \hline
vc\_20\_120\_01.txt   & 910            & 844          & {[}0, 1, 3,...,19{]}   & 134,8148148 & 3,625911951 \\ \hline
vc\_20\_120\_02.txt   & 1086           & 1009         & {[}0, 1, 3,...,19{]}   & 160,8888889 & 3,825074911 \\ \hline
vc\_20\_120\_03.txt   & 1039           & 994          & {[}0, 1, 2,...,19{]}   & 153,9259259 & 3,100630999 \\ \hline
vc\_20\_120\_04.txt   & 1097           & 1050         & {[}0, 1, 2,...,19{]}   & 162,5185185 & 3,225299835 \\ \hline
vc\_20\_120\_05.txt   & 1029           & 997          & {[}1, 3, 4,...,19{]}   & 152,4444444 & 3,18448782  \\ \hline
vc\_20\_120\_06.txt   & 961            & 961          & {[}0, 1, 2,...,19{]}   & 142,3703704 & 3,598522902 \\ \hline
vc\_20\_120\_07.txt   & 991            & 991          & {[}0, 1, 2,...,19{]}   & 146,8148148 & 4,857724905 \\ \hline
vc\_20\_120\_08.txt   & 1170           & 1142         & {[}1, 2, 3,...,19{]}   & 173,3333333 & 3,927416086 \\ \hline
vc\_20\_120\_09.txt   & 1296           & 1261         & {[}0, 1, 2,...,19{]}   & 192         & 3,158335924 \\ \hline
vc\_20\_120\_10.txt   & 1148           & 1133         & {[}0, 2, 4,...,19{]}   & 170,0740741 & 3,382470131 \\ \hline
vc\_25\_150\_01.txt   & 1463           & 1312         & {[}0, 2, 3,..., 24{]}  & 216,7407407 & 4,598255157 \\ \hline
vc\_25\_150\_02.txt   & 1234           & 1132         & {[}1, 2, 3,...,24{]}   & 182,8148148 & 4,611696959 \\ \hline
vc\_25\_150\_03.txt   & 1416           & 1352         & {[}0, 1, 2,...,24{]}   & 209,7777778 & 4,227385044 \\ \hline
vc\_25\_150\_04.txt   & 1519           & 1425         & {[}0, 1, 2,...,24{]}   & 225,037037  & 4,095405102 \\ \hline
vc\_25\_150\_05.txt   & 1390           & 1307         & {[}0, 2, 3,...,24{]}   & 205,9259259 & 4,551862001 \\ \hline
vc\_25\_150\_06.txt   & 1328           & 1255         & {[}0, 1, 2,...,22{]}   & 196,7407407 & 3,764107227 \\ \hline
vc\_25\_150\_07.txt   & 1524           & 1511         & {[}0, 1, 4,...,24{]}   & 225,7777778 & 4,488750696 \\ \hline
vc\_25\_150\_08.txt   & 1151           & 1151         & {[}0, 1, 2,...,24{]}   & 170,5185185 & 4,198129177 \\ \hline
vc\_25\_150\_09.txt   & 1248           & 1248         & {[}1, 2, 3,...,24{]}   & 184,8888889 & 4,251717091 \\ \hline
vc\_25\_150\_10.txt   & 1061           & 1061         & {[}0, 1, 2,...,23{]}   & 157,1851852 & 4,223310947 \\ \hline
vc\_100\_500\_01.txt  & 4878           & 4597         & {[}1, 2, 4,...,99{]}   & 68,64529305 & 133,8012891 \\ \hline
vc\_100\_500\_02.txt  & 5273           & 5033         & {[}1, 3, 4,...,99{]}   & 43,84079923 & 110,9794841 \\ \hline
vc\_100\_500\_03.txt  & 4592           & 4424         & {[}1, 3, 4,...,99{]}   & 39,26773973 & 83,4767592  \\ \hline
vc\_100\_500\_04.txt  & 4980           & 4636         & {[}2, 6, 8,...,99{]}   & 67,07962836 & 91,55720091 \\ \hline
vc\_100\_500\_05.txt  & 5057           & 4885         & {[}1, 3, 4,...,97{]}   & 60,93683859 & 106,0464609 \\ \hline
vc\_100\_500\_06.txt  & 5069           & 4860         & {[}0, 2, 3,...,99{]}   & 98,89224204 & 67,52395678 \\ \hline
vc\_100\_500\_07.txt  & 5082           & 4694         & {[}0, 1, 2,...,99{]}   & 99,14586191 & 41,51749468 \\ \hline
vc\_100\_500\_08.txt  & 4814           & 4673         & {[}0, 1, 2,...,98{]}   & 93,91739064 & 47,34881878 \\ \hline
vc\_100\_500\_09.txt  & 4722           & 4657         & {[}1, 2, 3,...,98{]}   & 92,1225423  & 53,81567311 \\ \hline
vc\_100\_500\_10.txt  & 4601           & 4515         & {[}0, 1, 2,...,98{]}   & 89,76192654 & 46,76840401 \\ 
\end{tabular}
\end{table}
\pagebreak

\begin{table}[]
\begin{tabular}{|l|l|l|l|l|l|}
Input name            & Initial Weight & Final Weight & Final Solution         & Final Eps   & Run Time    \\ \hline
vc\_100\_2000\_01.txt & 6264           & 6111         & {[}0, 1, 2,...99{]}    & 122,2057613 & 194,4893651 \\ \hline
vc\_100\_2000\_02.txt & 6023           & 5992         & {[}0, 1, 2,...99{]}    & 117,504039  & 431,6642709 \\ \hline
vc\_100\_2000\_03.txt & 5830           & 5738         & {[}0, 1, 2,...,99{]}   & 113,7387593 & 261,161907  \\ \hline
vc\_100\_2000\_04.txt & 6145           & 6058         & {[}0, 1, 2,...,99{]}   & 72,10912012 & 477,8304288 \\ \hline
vc\_100\_2000\_05.txt & 6471           & 6330         & {[}0, 1, 3,...,99{]}   & 53,45673204 & 278,957222  \\ \hline
vc\_100\_2000\_06.txt & 5960           & 5904         & {[}0, 1, 2,...,99{]}   & 174,4124371 & 145,7328162 \\ \hline
vc\_100\_2000\_07.txt & 6504           & 6354         & {[}1, 2, 3,...,99{]}   & 126,8879744 & 136,9901099 \\ \hline
vc\_100\_2000\_08.txt & 6360           & 6309         & {[}0, 1, 2,...,99{]}   & 124,0786465 & 168,2622838 \\ \hline
vc\_100\_2000\_09.txt & 6001           & 5985         & {[}0, 1, 2,...,99{]}   & 117,0748362 & 157,6364491 \\ \hline
vc\_100\_2000\_10.txt & 6287           & 6240         & {[}2, 3, 4,...,99{]}   & 122,6544734 & 239,845624  \\ \hline
vc\_200\_750\_01.txt  & 9076           & 8603         & {[}0, 1, 2,...198{]}   & 229,1767746 & 223,5511329 \\ \hline
vc\_200\_750\_02.txt  & 8696           & 8384         & {[}0, 3, 4,...,199{]}  & 84,50493346 & 239,3247552 \\ \hline
vc\_200\_750\_03.txt  & 8952           & 8654         & {[}0, 1, 2,...,199{]}  & 261,9698217 & 132,4322062 \\ \hline
vc\_200\_750\_04.txt  & 8466           & 8074         & {[}1, 2, 3,...,199{]}  & 69,60856611 & 211,2878647 \\ \hline
vc\_200\_750\_05.txt  & 9173           & 9006         & {[}0, 1, 2,...,199{]}  & 151,3027671 & 223,7731762 \\ \hline
vc\_200\_750\_06.txt  & 8594           & 8265         & {[}0, 1, 3,...,197{]}  & 33,11846847 & 176,7502    \\ \hline
vc\_200\_750\_07.txt  & 8005           & 7701         & {[}1, 3, 4,...,199{]}  & 156,1713153 & 135,1315081 \\ \hline
vc\_200\_750\_08.txt  & 8712           & 8528         & {[}0, 3, 4,...,199{]}  & 57,91196203 & 240,9734781 \\ \hline
vc\_200\_750\_09.txt  & 8830           & 8661         & {[}0, 8, 10,...,199{]} & 157,7677004 & 248,000886  \\ \hline
vc\_200\_750\_10.txt  & 8708           & 8459         & {[}1, 2, 4,...,199{]}  & 169,8862978 & 133,300956  \\ \hline
vc\_200\_3000\_01.txt & 11895          & 11728        & {[}0, 2, 3,...,199{]}  & 197,980836  & 710,657182  \\ \hline
vc\_200\_3000\_02.txt & 11576          & 11346        & {[}0, 1, 2,...,199{]}  & 131,1731888 & 623,197376  \\ \hline
vc\_200\_3000\_03.txt & 12227          & 12030        & {[}2, 3, 4,...,199{]}  & 20,94171716 & 678,0509641 \\ \hline
vc\_200\_3000\_04.txt & 13136          & 12894        & {[}0, 2, 3,...,199{]}  & 175,0668067 & 690,9683321 \\ \hline
vc\_200\_3000\_05.txt & 11950          & 11858        & {[}0, 1, 2,...,199{]}  & 202,771632  & 696,4376061 \\ \hline
vc\_200\_3000\_06.txt & 11539          & 11328        & {[}0, 1, 3,...,199{]}  & 66,169563   & 691,5875759 \\ \hline
vc\_200\_3000\_07.txt & 11836          & 11583        & {[}1, 2, 3,...,199{]}  & 214,2054142 & 659,8127627 \\ \hline
vc\_200\_3000\_08.txt & 12083          & 11868        & {[}1, 2, 3,...,199{]}  & 231,7402946 & 679,6931341 \\ \hline
vc\_200\_3000\_09.txt & 12118          & 12008        & {[}0, 1, 2,...,199{]}  & 200,4280632 & 692,4992981 \\ \hline
vc\_200\_3000\_10.txt & 11915          & 11551        & {[}0, 1, 2,...,198{]}  & 168,2991086 & 689,6815619 \\ \hline
vc\_800\_10000.txt    & 46155          & 44649        & {[}0, 2, 3,...,799{]}  & 400,4481024 & 348,5432  
\end{tabular}
\end{table}


\pagebreak

\qquad

\qquad

\qquad

\qquad

\qquad

\qquad

\qquad

\qquad

\qquad

\qquad

\qquad


\section{Conclusion}

\qquad

\qquad

This project has made possible the acquisition of several practical skills analyzed during the course of Artificial Intelligence and has allowed to become aware of the use and development of solutions based on these algorithms.

The results obtained are quite good since randomness makes it possible to make locally optimal choices, especially thanks to the greedy algorithm that builds a good starting solution. The values of the parameters that have been chosen after many attempts are also quite good. There is definitely room for improvement in the sampling of neighbors, you could think of new selection criteria.   

\qquad

\qquad




\pagebreak



\end{document}  




